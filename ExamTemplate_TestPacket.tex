\documentclass{scioly-exam-template-araneesh}

\newcommand{\printdoctype}{Test Packet}

\begin{document}

% == Test Packet Title ==
\begin{center}
\vspace*{1em}
{\Huge \textbf{
        %Science Olympiad\\
        \printtournament\\
        \printdate\\
        \vspace{4mm}
    }
}
\begin{figure}[h]
    \centering
    \includegraphics[width=4in,height=3in,keepaspectratio]{images/logo.png}
\end{figure}
{\Huge \textbf{
        \printevent\\
        \printdoctype\\
        \vspace{4mm}
    }
}
{\Large \textbf{\printauthor} \\ \vspace{4mm}}
\end{center}

% == Instructions ==
% Adjust as needed, of course
\begin{large}
\begin{center} \textbf{\Large Instructions} \end{center}
\begin{enumerate}
    \itemsep-0.25em
    \item You will have 50 minutes to take this test.
    \item Answer all questions on your Answer Sheet. The space provided for each question is sufficient to fit a response earning full points.
    \item All questions have their value indicated in parentheses, both in this Test Packet and on the Answer Sheet. This test has a total of \sumpoints~points.
%    \item Do not write on or take apart the pages of the Test Packet or Image Sheet.
    \item You are free to write on and take apart the pages of the Test Packet or Image Sheet as you wish.
    \item For numerical questions, answers within a reasonable range will earn full credit - accounting for significant figures is not required.
% Event-specific instructions (leaving my standard Geologic Mapping ones in here as examples):
%    \item Give answers for strike and dip using the right-hand rule, i.e. choose the strike value such that the dip direction is 90\degree clockwise from it (rather than the one where the dip direction is 90\degree counterclockwise from it).
%    \item For maps, assume that images are oriented with due north toward the top of the image number, unless informed otherwise.
    \item Ties will be broken by comparing points earned for each question in order.
\end{enumerate}
\end{large}
\pagebreak

% == Exam Content ==
\begin{multicols}{2}
\begin{enumerate}
\begin{large}

% Iterate over data and format entries as appropriate for this doctype
\newcounter{datarow}\forloop{datarow}{1}{\value{datarow} < \numexpr\sciolydataarrayROWS + 1\relax}{\formatDataEntryTestpacket{\thedatarow}}

\vfill\null
\label{LastPageRef}

\end{large}
\end{enumerate}
\end{multicols}

\end{document}
